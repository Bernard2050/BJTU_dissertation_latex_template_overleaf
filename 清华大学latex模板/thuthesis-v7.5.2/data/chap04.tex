% !TeX root = ../thuthesis-example.tex

\chapter{引用文献的标注}

模板支持 BibTeX 和 BibLaTeX 两种方式处理参考文献。
下文主要介绍 BibTeX 配合 \pkg{natbib} 宏包的主要使用方法。


\section{顺序编码制}

在顺序编码制下,默认的 \cs{cite} 命令同 \cs{citep} 一样,序号置于方括号中,
引文页码会放在括号外。
统一处引用的连续序号会自动用短横线连接。

\thusetup{
  cite-style = super,
}
\noindent
\begin{tabular}{l@{\quad$\Rightarrow$\quad}l}
  \verb|\cite{zhangkun1994}|               & \cite{zhangkun1994}               \\
  \verb|\citet{zhangkun1994}|              & \citet{zhangkun1994}              \\
  \verb|\citep{zhangkun1994}|              & \citep{zhangkun1994}              \\
  \verb|\cite[42]{zhangkun1994}|           & \cite[42]{zhangkun1994}           \\
  \verb|\cite{zhangkun1994,zhukezhen1973}| & \cite{zhangkun1994,zhukezhen1973} \\
\end{tabular}


也可以取消上标格式,将数字序号作为文字的一部分。
建议全文统一使用相同的格式。

\thusetup{
  cite-style = inline,
}
\noindent
\begin{tabular}{l@{\quad$\Rightarrow$\quad}l}
  \verb|\cite{zhangkun1994}|               & \cite{zhangkun1994}               \\
  \verb|\citet{zhangkun1994}|              & \citet{zhangkun1994}              \\
  \verb|\citep{zhangkun1994}|              & \citep{zhangkun1994}              \\
  \verb|\cite[42]{zhangkun1994}|           & \cite[42]{zhangkun1994}           \\
  \verb|\cite{zhangkun1994,zhukezhen1973}| & \cite{zhangkun1994,zhukezhen1973} \\
\end{tabular}



\section{著者-出版年制}

著者-出版年制下的 \cs{cite} 跟 \cs{citet} 一样。

\thusetup{
  cite-style = author-year,
}
\noindent
\begin{tabular}{@{}l@{$\Rightarrow$}l@{}}
  \verb|\cite{zhangkun1994}|                & \cite{zhangkun1994}                \\
  \verb|\citet{zhangkun1994}|               & \citet{zhangkun1994}               \\
  \verb|\citep{zhangkun1994}|               & \citep{zhangkun1994}               \\
  \verb|\cite[42]{zhangkun1994}|            & \cite[42]{zhangkun1994}            \\
  \verb|\citep{zhangkun1994,zhukezhen1973}| & \citep{zhangkun1994,zhukezhen1973} \\
\end{tabular}

\vskip 2ex
\thusetup{
  cite-style = super,
}
注意,引文参考文献的每条都要在正文中标注
\cite{zhangkun1994,zhukezhen1973,dupont1974bone,zhengkaiqing1987,%
  jiangxizhou1980,jianduju1994,merkt1995rotational,mellinger1996laser,%
  bixon1996dynamics,mahui1995,carlson1981two,taylor1983scanning,%
  taylor1981study,shimizu1983laser,atkinson1982experimental,%
  kusch1975perturbations,guangxi1993,huosini1989guwu,wangfuzhi1865songlun,%
  zhaoyaodong1998xinshidai,biaozhunhua2002tushu,chubanzhuanye2004,%
  who1970factors,peebles2001probability,baishunong1998zhiwu,%
  weinstein1974pathogenic,hanjiren1985lun,dizhi1936dizhi,%
  tushuguan1957tushuguanxue,aaas1883science,fugang2000fengsha,%
  xiaoyu2001chubanye,oclc2000about,scitor2000project%
}。
